\documentclass{article}
\usepackage[utf8]{inputenc}
\usepackage{graphicx}
\usepackage[francais]{babel}
\usepackage{placeins}
\usepackage[top=3cm, bottom=5cm, left=3cm , right=3cm]{geometry}
\usepackage{pdfpages}

\usepackage{amsmath}
\usepackage[]{algorithm2e}
\usepackage[table]{xcolor}
\usepackage{tabularx}
\usepackage{array}
\usepackage[margin=3cm]{geometry}
\usepackage{color}
\usepackage{url}


\newcommand{\hsp}{\hspace{20pt}}
\newcommand{\HRule}{\rule{\linewidth}{0.5mm}} 


\title{(Projet RS-système)Rapport_ASSELIN-Bo}
\author{charlotte.asselin-boulle }
\date{May 2017}



\makeindex

\begin{document}

\begin{titlepage}
\begin{minipage}[c]{5.5cm}       \includegraphics[width=3cm]{images/LogoTelecom.jpg}    \end{minipage}
\begin{minipage}[c]{5.5cm}       \includegraphics[width=3.5cm]{images/Lorraine.png}    \end{minipage}   
\begin{minipage}[c]{5.5cm}       \includegraphics[width=3.5cm]{images/logo_universite.png}    \end{minipage}

\bigbreak
  \begin{sffamily}
  \begin{center}
    \textsc{Telecom Nancy - deuxième année}\\[0.3cm]
    \medbreak\textsc{\Large Module RS : Réseaux et Systèmes}\\[0,2cm]
    
    \HRule \\[0,3cm]
     \Huge\textbf{\textsc{Rapport du Projet de Systèmes }}
    \medbreak
    \LARGE \textsc{tesh}
    \bigbreak
    \bigbreak
    \bigbreak
    \large \textsc{ASSELIN-BOULLÉ Charlotte et Valentin MOREAU}
     \bigbreak
     \bigbreak
    À rendre le Vendredi 22 Décembre 2017
    \bigbreak
    \HRule \\[0,6cm]
   
    
    \end{center}
    
  \end{sffamily}
\end{titlepage}


  

%##########################################################################
%##########################################################################
%   SOMMAIRE
%##########################################################################
%##########################################################################

\newpage

\begin{centering}
\large \tableofcontents
\end{centering}


%##########################################################################
%##########################################################################
%   INTRODUCTION
%##########################################################################
%##########################################################################
\newpage
\vspace*{2cm}
\section{Introduction}
\vspace*{0,5cm}
%TODO 
%contient nos choix de conception, nos difficultés et leur résolutions (éventuelles) ainsi que la répartition de travail au sein de notre groupe et un bilan personnel. 


\section{Remerciements}
Pour leur aide précieuses nous remercions : \\
%Monsieur Sébastien DA SILVA, professeur à TELECOM Nancy, qui ...
%Monsieur Baptiste AURIAC, élève à TELECOM Nancy en deuxième année qui ...
%Messieurs Jean CALVET et Clément MARTINEZ, élèves à TELECOM Nancy en deuxième année qui ...




%##########################################################################
%##########################################################################
%   ÉTUDE DE L'EXISTANT
%##########################################################################
%##########################################################################
\newpage
\section{Étude de l'art}

%TODO 
%\subsection{ }


%##########################################################################
%##########################################################################
%   Notre projet 
%##########################################################################
%##########################################################################
\newpage
\section{Notre projet}
Cette partie est la présentation des différentes parties avec les difficultés rencontrées et manières de les surmonter.


\subsection{Général}

Ne pas oublier de mettre à jour le Makefile ou encore d'inclure les fichiers nécessaires (avec le chemin d'accès). Erreur d'étourderie qui nous valait de nous arracher les cheveux au moment de compiler. 


\subsection{Fonctionnement de base}
Le tesh doit attendre une commande l'exécuter et rester en attente. Nous avons donc choisis que le programme attende le signal de la touche "entrée" et qu'il exécute la commande dans le fils d'un fork, le père étant le tesh global qui ne se ferme que avec le signal de fin "crtl+c". 

\subsection{Gestion des arguments et enchaînement conditionnel de commandes et redirection }
% cm1 ; cm2 (1 puis 2)      cm1 & cm2 (1 puis si code 0, 2)     cm1 || cm2 (1 puis si pas code 0, 2) 
% >     >>      | 
% test : cmd1 < fichier1 | cmd2 | cmd3 >> fichier2
\subsection{Commande interne}
%cd : 
%La commande cd doit être traitée de manière particulière, car, si elle était exécutée dans un processus fils, son effet serait limité à ce processus fils, et n’aurait pas d’effet sur la commande suivante. Il faut donc l’implémenter comme une commande interne du shell (on parle de built-in).

\subsection{Gestion des répertoires courants et prompts}
%Affichage d’une invite de commande (prompt). En mode interactif, le shell affiche un prompt de la forme ”USER@HOSTNAME:REPCOURANT$ ”. USER est le nom de l’utilisateur courant, HOSTNAME est l’hostname de la machine (cf gethostname(2)), et REPCOURANT est le répertoire courant.

\subsection{Mode non itératif }
%si un fichier contenant les commandes à exécuter est passé en paramètre (./tesh monscripttesh), soit si l’entrée standard de tesh n’est pas un terminal (à tester avec isatty(3)). Quand tesh s’exécute en mode non-interactif, alors il n’affiche pas de prompt.

\subsection{Gestion des erreurs}
% si exécuter avec -e alors tesh s’arrête dès qu’une commande
termine avec un code de retour différent de 0 (c’est particulièrement utile en mode non-interactif).

\subsection{gestion de différents processus}
%commandes en arrière plan, fg 
%Si une commande est suivie du séparateur de commandes &, alors elle doit être lancée en arrière plan. Le shell doit simplement afficher son PID, sous la forme [pid] (par exemple [42]). Ensuite, une commande interne fg doit permettre de ramener la commande au premier plan et d’en attendre la fin. Si fg est appelé sans paramètre, alors le shell attend la fin d’un des processus en arrière plan. Si un PID est passé en paramètre, alors le shell attend la fin de ce processus. Dans tous les cas, le shell doit afficher le PID du processus en arrière plan qui a terminé, et son code de retour, sous la forme [pid->retcode] (par exemple [42->2], si le processus 42 a terminé avec le code de retour 2.
\subsection{...}
%Edition de la ligne de commande du shell avec readline. Si l’option -r est passée au shell, alors le shell doit utilisée la bibliothèque readline pour permettre l’édition interactive de la ligne de commande, et la gestion de l’historique.
%Chargement dynamique de la bibliothèque readline. Au lieu de lier le programme avec la bibliothèque readline, utilisez dlopen pour charger dynamiquement cette bibliothèque.

\subsection{Idées d'amélioration à apporter}
Le projet se voulait déjà très ambitieux mais nous avons eu le loisir de songer à des améliorations possibles : 
\begin{itemize}
    \item 
    \item etc... 
\end{itemize}

%##########################################################################
%##########################################################################
%   GESTION DE PROJET
%##########################################################################
%##########################################################################
\newpage
\section{Gestion de projet}


%\\ Pour une question d'habitude et nos avons utilisé GitHub et installé un mirroiring sur la Forge (puisqu'il fallait qu'il y ait une activité et un rendu sur la forge). 
%\\ Enfin nous avons beaucoup programmer en groupe pour s'entraider et puis partager nos idées car avoir un double voire triple regard sur la même chose s'avoue souvent bénéfique.  


\subsection{Répartition du travail et temps passé sur les parties}

Le tableau ci-dessous représente la responsabilité et surtout le temps de travail de notre groupe sur chaque partie (considéré comme des jâlons. 

\begin{center}
\begin{tabular}{| p {10cm} || m{2cm} | m{2cm} | m{2cm} |}
  \hline
  Tâches            & Charlotte     & Valentin \\
  \hline 
  \hline
    Gestion de Projet (Rapport, gestion du git, études de l'art) 
                    &  1h           & .h 
    \\ \hline 
    Fonctionnement de Base et fonctionnalités globales 
                    &  1h           & .h 
    \\ \hline 


  
\end{tabular}
\caption{\textid{\\Matrice %RACI 
représentant notre répartition du travail avec le temps passé \\ 
%\color{blue}R pour Responsable\color{black}, \color{red}A pour Acteur(e) \color{black},\color{green}C pour Consulté(e)\color{black} et I pour Informé(e) }}
\end{center}
\\

   
\subsection{Bilan personnel du projet }
%TODO 



%##########################################################################
%##########################################################################
%   CONCLUSION
%##########################################################################
%##########################################################################
\newpage
\vspace*{2cm}
\section{Conclusion}
\vspace*{0,5cm}
%TODO


\break
%##########################################################################
%##########################################################################
%   ANNEXES
%##########################################################################
%##########################################################################
\newpage
\section{Annexes}

\subsection{Sitographie}
%TODO 

\noindent
\textbf{[1] Titre }\\
\noindent
auteur, 
date publication  
\\
\noindent
\url{  }\\


%\noindent
%\textbf{[_] titre}\\
%\noindent
%Auteur\\
%\noindent
%date publication \\
%\noindent
%\url{http://   }\\



\subsection{Structure de notre programme}
%TODO CC des différents fichiers et leur liens pour la compilation


\end{document}

